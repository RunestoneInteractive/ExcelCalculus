%**************************************%
%* Generated from MathBook XML source *%
%*    on 2017-03-20T13:55:01-05:00    *%
%*                                    *%
%*   http://mathbook.pugetsound.edu   *%
%*                                    *%
%**************************************%
\documentclass[10pt,]{book}
%% Custom Preamble Entries, early (use latex.preamble.early)
%% Inline math delimiters, \(, \), need to be robust
%% 2016-01-31:  latexrelease.sty  supersedes  fixltx2e.sty
%% If  latexrelease.sty  exists, bugfix is in kernel
%% If not, bugfix is in  fixltx2e.sty
%% See:  https://tug.org/TUGboat/tb36-3/tb114ltnews22.pdf
%% and read "Fewer fragile commands" in distribution's  latexchanges.pdf
\IfFileExists{latexrelease.sty}{}{\usepackage{fixltx2e}}
%% Text height identically 9 inches, text width varies on point size
%% See Bringhurst 2.1.1 on measure for recommendations
%% 75 characters per line (count spaces, punctuation) is target
%% which is the upper limit of Bringhurst's recommendations
%% Load geometry package to allow page margin adjustments
\usepackage{geometry}
\geometry{letterpaper,total={340pt,9.0in}}
%% Custom Page Layout Adjustments (use latex.geometry)
%% This LaTeX file may be compiled with pdflatex, xelatex, or lualatex
%% The following provides engine-specific capabilities
%% Generally, xelatex and lualatex will do better languages other than US English
%% You can pick from the conditional if you will only ever use one engine
\usepackage{ifthen}
\usepackage{ifxetex,ifluatex}
\ifthenelse{\boolean{xetex} \or \boolean{luatex}}{%
%% begin: xelatex and lualatex-specific configuration
%% fontspec package will make Latin Modern (lmodern) the default font
\ifxetex\usepackage{xltxtra}\fi
\usepackage{fontspec}
%% realscripts is the only part of xltxtra relevant to lualatex 
\ifluatex\usepackage{realscripts}\fi
%% 
%% Extensive support for other languages
\usepackage{polyglossia}
\setdefaultlanguage{english}
%% Magyar (Hungarian)
\setotherlanguage{magyar}
%% Spanish
\setotherlanguage{spanish}
%% Vietnamese
\setotherlanguage{vietnamese}
%% end: xelatex and lualatex-specific configuration
}{%
%% begin: pdflatex-specific configuration
%% translate common Unicode to their LaTeX equivalents
%% Also, fontenc with T1 makes CM-Super the default font
%% (\input{ix-utf8enc.dfu} from the "inputenx" package is possible addition (broken?)
\usepackage[T1]{fontenc}
\usepackage[utf8]{inputenc}
%% end: pdflatex-specific configuration
}
%% Monospace font: Inconsolata (zi4)
%% Sponsored by TUG: http://levien.com/type/myfonts/inconsolata.html
%% See package documentation for excellent instructions
%% One caveat, seem to need full file name to locate OTF files
%% Loads the "upquote" package as needed, so we don't have to
%% Upright quotes might come from the  textcomp  package, which we also use
%% We employ the shapely \ell to match Google Font version
%% pdflatex: "varqu" option produces best upright quotes
%% xelatex,lualatex: add StylisticSet 1 for shapely \ell
%% xelatex,lualatex: add StylisticSet 2 for plain zero
%% xelatex,lualatex: we add StylisticSet 3 for upright quotes
%% 
\ifthenelse{\boolean{xetex} \or \boolean{luatex}}{%
%% begin: xelatex and lualatex-specific monospace font
\usepackage{zi4}
\setmonofont[BoldFont=Inconsolatazi4-Bold.otf,StylisticSet={1,3}]{Inconsolatazi4-Regular.otf}
%% end: xelatex and lualatex-specific monospace font
}{%
%% begin: pdflatex-specific monospace font
\usepackage[varqu]{zi4}
%% end: pdflatex-specific monospace font
}
%% Symbols, align environment, bracket-matrix
\usepackage{amsmath}
\usepackage{amssymb}
%% allow more columns to a matrix
%% can make this even bigger by overriding with  latex.preamble.late  processing option
\setcounter{MaxMatrixCols}{30}
%%
%% Color support, xcolor package
%% Always loaded.  Used for:
%% mdframed boxes, add/delete text, author tools
\PassOptionsToPackage{usenames,dvipsnames,svgnames,table}{xcolor}
\usepackage{xcolor}
%%
%% Semantic Macros
%% To preserve meaning in a LaTeX file
%% Only defined here if required in this document
%% Subdivision Numbering, Chapters, Sections, Subsections, etc
%% Subdivision numbers may be turned off at some level ("depth")
%% A section *always* has depth 1, contrary to us counting from the document root
%% The latex default is 3.  If a larger number is present here, then
%% removing this command may make some cross-references ambiguous
%% The precursor variable $numbering-maxlevel is checked for consistency in the common XSL file
\setcounter{secnumdepth}{3}
%% Environments with amsthm package
%% Theorem-like environments in "plain" style, with or without proof
\usepackage{amsthm}
\theoremstyle{plain}
%% Numbering for Theorems, Conjectures, Examples, Figures, etc
%% Controlled by  numbering.theorems.level  processing parameter
%% Always need a theorem environment to set base numbering scheme
%% even if document has no theorems (but has other environments)
\newtheorem{theorem}{Theorem}[section]
%% Only variants actually used in document appear here
%% Style is like a theorem, and for statements without proofs
%% Numbering: all theorem-like numbered consecutively
%% i.e. Corollary 4.3 follows Theorem 4.2
%% Miscellaneous environments, normal text
%% Numbering for inline exercises and lists is in sync with theorems, etc
\theoremstyle{definition}
\newtheorem{exercise}[theorem]{Exercise}
%% Localize LaTeX supplied names (possibly none)
\renewcommand*{\appendixname}{Appendix}
\renewcommand*{\chaptername}{Chapter}
%% Raster graphics inclusion, wrapped figures in paragraphs
%% \resizebox sometimes used for images in side-by-side layout
\usepackage{graphicx}
%%
%% Program listing support, for inline code, Sage code
\usepackage{listings}
%% We define the listings font style to be the default "ttfamily"
%% To fix hyphens/dashes rendered in PDF as fancy minus signs by listing
%% http://tex.stackexchange.com/questions/33185/listings-package-changes-hyphens-to-minus-signs
\makeatletter
\lst@CCPutMacro\lst@ProcessOther {"2D}{\lst@ttfamily{-{}}{-{}}}
\@empty\z@\@empty
\makeatother
\ifthenelse{\boolean{xetex}}{}{%
%% begin: pdflatex-specific listings configuration
%% translate U+0080 - U+00F0 to their textmode LaTeX equivalents
%% Data originally from https://www.w3.org/Math/characters/unicode.xml, 2016-07-23
%% Lines marked in XSL with "$" were converted from mathmode to textmode
\lstset{extendedchars=true}
\lstset{literate={ }{{~}}{1}{¡}{{\textexclamdown }}{1}{¢}{{\textcent }}{1}{£}{{\textsterling }}{1}{¤}{{\textcurrency }}{1}{¥}{{\textyen }}{1}{¦}{{\textbrokenbar }}{1}{§}{{\textsection }}{1}{¨}{{\textasciidieresis }}{1}{©}{{\textcopyright }}{1}{ª}{{\textordfeminine }}{1}{«}{{\guillemotleft }}{1}{¬}{{\textlnot }}{1}{­}{{\-}}{1}{®}{{\textregistered }}{1}{¯}{{\textasciimacron }}{1}{°}{{\textdegree }}{1}{±}{{\textpm }}{1}{²}{{\texttwosuperior }}{1}{³}{{\textthreesuperior }}{1}{´}{{\textasciiacute }}{1}{µ}{{\textmu }}{1}{¶}{{\textparagraph }}{1}{·}{{\textperiodcentered }}{1}{¸}{{\c{}}}{1}{¹}{{\textonesuperior }}{1}{º}{{\textordmasculine }}{1}{»}{{\guillemotright }}{1}{¼}{{\textonequarter }}{1}{½}{{\textonehalf }}{1}{¾}{{\textthreequarters }}{1}{¿}{{\textquestiondown }}{1}{À}{{\`{A}}}{1}{Á}{{\'{A}}}{1}{Â}{{\^{A}}}{1}{Ã}{{\~{A}}}{1}{Ä}{{\"{A}}}{1}{Å}{{\AA }}{1}{Æ}{{\AE }}{1}{Ç}{{\c{C}}}{1}{È}{{\`{E}}}{1}{É}{{\'{E}}}{1}{Ê}{{\^{E}}}{1}{Ë}{{\"{E}}}{1}{Ì}{{\`{I}}}{1}{Í}{{\'{I}}}{1}{Î}{{\^{I}}}{1}{Ï}{{\"{I}}}{1}{Ð}{{\DH }}{1}{Ñ}{{\~{N}}}{1}{Ò}{{\`{O}}}{1}{Ó}{{\'{O}}}{1}{Ô}{{\^{O}}}{1}{Õ}{{\~{O}}}{1}{Ö}{{\"{O}}}{1}{×}{{\texttimes }}{1}{Ø}{{\O }}{1}{Ù}{{\`{U}}}{1}{Ú}{{\'{U}}}{1}{Û}{{\^{U}}}{1}{Ü}{{\"{U}}}{1}{Ý}{{\'{Y}}}{1}{Þ}{{\TH }}{1}{ß}{{\ss }}{1}{à}{{\`{a}}}{1}{á}{{\'{a}}}{1}{â}{{\^{a}}}{1}{ã}{{\~{a}}}{1}{ä}{{\"{a}}}{1}{å}{{\aa }}{1}{æ}{{\ae }}{1}{ç}{{\c{c}}}{1}{è}{{\`{e}}}{1}{é}{{\'{e}}}{1}{ê}{{\^{e}}}{1}{ë}{{\"{e}}}{1}{ì}{{\`{\i}}}{1}{í}{{\'{\i}}}{1}{î}{{\^{\i}}}{1}{ï}{{\"{\i}}}{1}{ð}{{\dh }}{1}{ñ}{{\~{n}}}{1}{ò}{{\`{o}}}{1}{ó}{{\'{o}}}{1}{ô}{{\^{o}}}{1}{õ}{{\~{o}}}{1}{ö}{{\"{o}}}{1}{÷}{{\textdiv }}{1}{ø}{{\o }}{1}{ù}{{\`{u}}}{1}{ú}{{\'{u}}}{1}{û}{{\^{u}}}{1}{ü}{{\"{u}}}{1}{ý}{{\'{y}}}{1}{þ}{{\th }}{1}{ÿ}{{\"{y}}}{1}}
%% end: pdflatex-specific listings configuration
}
%% End of generic listing adjustments
%% Sage's blue is 50%, we go way lighter (blue!05 would work)
\definecolor{sageblue}{rgb}{0.95,0.95,1}
%% Sage input, listings package: Python syntax, boxed, colored, line breaking
%% Indent from left margin, flush at right margin
\lstdefinestyle{sageinput}{language=Python,breaklines=true,breakatwhitespace=true,basicstyle=\small\ttfamily,columns=fixed,frame=single,backgroundcolor=\color{sageblue},xleftmargin=4ex}
%% Sage output, similar, but not boxed, not colored
\lstdefinestyle{sageoutput}{language=Python,breaklines=true,breakatwhitespace=true,basicstyle=\small\ttfamily,columns=fixed,xleftmargin=4ex}
%% More flexible list management, esp. for references and exercises
%% But also for specifying labels (i.e. custom order) on nested lists
\usepackage{enumitem}
%% Lists of exercises in their own section, maximum depth 4
\newlist{exerciselist}{description}{4}
\setlist[exerciselist]{leftmargin=0pt,itemsep=1.0ex,topsep=1.0ex,partopsep=0pt,parsep=0pt}
%% Support for index creation
%% imakeidx package does not require extra pass (as with makeidx)
%% Title of the "Index" section set via a keyword
%% Language support for the "see" and "see also" phrases
\usepackage{imakeidx}
\makeindex[title=Index, intoc=true]
\renewcommand{\seename}{see}
\renewcommand{\alsoname}{see also}
%% hyperref driver does not need to be specified
\usepackage{hyperref}
%% configure hyperref's  \url  to match listings' inline verbatim
\renewcommand\UrlFont{\small\ttfamily}
%% Hyperlinking active in PDFs, all links solid and blue
\hypersetup{colorlinks=true,linkcolor=blue,citecolor=blue,filecolor=blue,urlcolor=blue}
\hypersetup{pdftitle={Business Calculus with Excel}}
%% If you manually remove hyperref, leave in this next command
\providecommand\phantomsection{}
%% If tikz has been loaded, replace ampersand with \amp macro
%% extpfeil package for certain extensible arrows,
%% as also provided by MathJax extension of the same name
%% NB: this package loads mtools, which loads calc, which redefines
%%     \setlength, so it can be removed if it seems to be in the 
%%     way and your math does not use:
%%     
%%     \xtwoheadrightarrow, \xtwoheadleftarrow, \xmapsto, \xlongequal, \xtofrom
%%     
%%     we have had to be extra careful with variable thickness
%%     lines in tables, and so also load this package late
\usepackage{extpfeil}
%% Custom Preamble Entries, late (use latex.preamble.late)
%% Begin: Author-provided packages
%% (From  docinfo/latex-preamble/package  elements)
%% End: Author-provided packages
%% Begin: Author-provided macros
%% (From  docinfo/macros  element)
%% Plus three from MBX for XML characters
\newcommand{\doubler}[1]{2#1}
\newcommand{\lt}{ < }
\newcommand{\gt}{ > }
\newcommand{\amp}{ & }
%% End: Author-provided macros
%% Title page information for book
\title{Business Calculus with Excel}
\author{Mike May, S.J.\\
Saint Louis UNiversity
}
\date{March 20, 2017}
\begin{document}
\frontmatter
%% begin: half-title
\thispagestyle{empty}
{\centering
\vspace*{0.28\textheight}
{\Huge Business Calculus with Excel}\\}
\clearpage
%% end:   half-title
%% begin: adcard
\thispagestyle{empty}
\null%
\clearpage
%% end:   adcard
%% begin: title page
%% Inspired by Peter Wilson's "titleDB" in "titlepages" CTAN package
\thispagestyle{empty}
{\centering
\vspace*{0.14\textheight}
%% Target for xref to top-level element is ToC
\addtocontents{toc}{\protect\hypertarget{MyExample}{}}
{\Huge Business Calculus with Excel}\\[3\baselineskip]
{\Large Mike May, S.J.}\\[0.5\baselineskip]
{\Large Saint Louis UNiversity}\\[3\baselineskip]
{\Large March 20, 2017}\\}
\clearpage
%% end:   title page
%% begin: copyright-page
\thispagestyle{empty}
\vspace*{\stretch{2}}
\vspace*{\stretch{1}}
\null\clearpage
%% end:   copyright-page
An attempt to build a chapter.%
%% begin: table of contents
%% Adjust Table of Contents
\setcounter{tocdepth}{1}
\renewcommand*\contentsname{Contents}
\tableofcontents
%% end:   table of contents
\mainmatter
\typeout{************************************************}
\typeout{Introduction  }
\typeout{************************************************}
This is an attempt to convert my book on business calculus to MATH XML format%
\typeout{************************************************}
\typeout{Chapter 1 First Chapter of Minimal Example}
\typeout{************************************************}
\chapter[{First Chapter of Minimal Example}]{First Chapter of Minimal Example}\label{chap-1-MinExample}
\typeout{************************************************}
\typeout{Section 1.1 Section 1.1 Linear Functions and models}
\typeout{************************************************}
\section[{Section 1.1 Linear Functions and models}]{Section 1.1 Linear Functions and models}\label{section-textual}
Section 1.1 Linear Functions and models%
\par
text of first section%
\typeout{************************************************}
\typeout{Exercises 1.1.1 Exercises 1.1 Linear Functions and models}
\typeout{************************************************}
\subsection[{Exercises 1.1 Linear Functions and models}]{Exercises 1.1 Linear Functions and models}\label{exercises-set-sec-1-1}
For problems 1-6, given two points in the \((q,p)\) plane and a value \(q_0\):%
\leavevmode%
\begin{enumerate}[label=(\alph*)]
\item\hypertarget{li-1}{}Find the slope of the line determined by the points.%
\item\hypertarget{li-2}{}Give the equation of the line determined by the points.%
\item\hypertarget{li-3}{}Give the value of \(p\) predicted for \(q_0\) by the line.%
\end{enumerate}
\begin{exerciselist}
\item[1.]\hypertarget{exercise-1}{} 
Points \((2,5)\) and \((6,17)\), with \(q_0=4\).
%
\par\smallskip
\par\smallskip
\noindent\textbf{Hint.}\hypertarget{hint-1}{}\quad
Find the slope and use the point-slope form%
\par\smallskip
\noindent\textbf{Solution.}\hypertarget{solution-1}{}\quad
\leavevmode%
\begin{enumerate}[label=(\alph*)]
\item\hypertarget{li-4}{}First find the slope: \(m=  \frac{\text{change in }p}{\text{change in }q}
=  \frac{17-5}{6-2}=\frac{12}{4}=3\)%
\item\hypertarget{li-5}{}Next we find the equation of the line. There are several ways to do this and two methods are outlined below.%
%
\begin{itemize}[label=\textbullet]
\item{}Method 1: use the point-slope equation: \(p-p_0=m (q-q_0)\).
We can choose either one of the points, so in this case we will find the line using the point \((q_0,p_0 )=(2,5)\). This gives the equation
\(p-5=3 (q-2)\).%
\par
Rewrite this as \(p=3q-1\)%
\item{}Method 2: use the slope- intercept equation \(p=m q+b\).
Use \((q,p)=(2,5)\) and \(m = 3\) and solve for \(b\):
\(5=3 (2)+b\).
And solving for \(b\) we have that \(b= -1\), and hence \(p=3q-1\)%
\end{itemize}
\item\hypertarget{li-8}{}Evaluate at the given point.  \(p(4)=3*4-1=11\)%
\end{enumerate}
\item[2.]\hypertarget{exercise-2}{} Points \((5,7)\) and \((10,7)\), with \(q_0=20\).
%
\par\smallskip
\item[3.]\hypertarget{exercise-3}{} Points \((20,10)\) and \((40,5)\), with \(q_0=12\).
%
\par\smallskip
\par\smallskip
\noindent\textbf{Solution.}\hypertarget{solution-2}{}\quad
Just as in problem 1 we find the slope and then find the equation of the line.%
\leavevmode%
\begin{enumerate}[label=(\alph*)]
\item\hypertarget{li-9}{}First find the slope: \(m=  \frac{\text{change in }p}{\text{change in }q}
=  \frac{5-10}{40-20}=-\frac{5}{20}=-\frac{1}{4}\)%
\item\hypertarget{li-10}{}Using \(p=m (q-q_0)+p_0\) with \((q_0,p_0 )=(20, 10)\) and \(m = -\frac{1}{4}\), we get \(p=-\frac{1}{4}(q-20)+10\).  Solving for \(p\) we get \(p =-\frac{1}{4}q+15\)%
\item\hypertarget{li-11}{}Evaluate at the given point.  \(p(12)=-\frac{1}{4}(12)+15=12\)%
\end{enumerate}
\item[4.]\hypertarget{exercise-4}{} Points \((5,62)\) and \((115,783)\), with \(q_0=415\).
%
\par\smallskip
\item[5.]\hypertarget{exercise-5}{} Points \((273,578)\) and \((412,6)\), with \(q_0=309\).
%
\par\smallskip
\par\smallskip
\noindent\textbf{Solution.}\hypertarget{solution-3}{}\quad
Just as in problem 1 we find the slope and then find the equation of the line.%
\leavevmode%
\begin{enumerate}[label=(\alph*)]
\item\hypertarget{li-12}{}First find the slope: \(m=  \frac{\text{change in }p}{\text{change in }q}
=  \frac{578-6}{273-412}=-\frac{5}{20}=-\frac{572}{139}\)%
\item\hypertarget{li-13}{}Using \(p=m (q-q_0)+p_0\) with \((q_0,p_0 )=(412, 6)\) and \(m = -\frac{572}{139}\), we get \(p=-\frac{572}{139}(q-412)+6\).  (We can combine the constant terms – the \(6\) and the \(-\frac{572}{139}*(-412)\), but leaving the equation in this form is acceptable.)%
\item\hypertarget{li-14}{}Evaluate at the given point.  \(p(309)=-\frac{572}{139}(309-412)+6
=-\frac{572}{139}(-103)+6=429\frac{119}{139}\)%
\end{enumerate}
\item[6.]\hypertarget{exercise-6}{} Points \((509,17)\) and \((211,132)\), with \(q_0=4\).
%
\par\smallskip
\item[7.]\hypertarget{exercise-7}{}\par\smallskip
\end{exerciselist}
\typeout{************************************************}
\typeout{Section 1.2 A Bit More Interesting}
\typeout{************************************************}
\section[{A Bit More Interesting}]{A Bit More Interesting}\label{section-interesting}
The previous section (\hyperref[section-textual]{Section~\ref{section-textual}}) was a bit boring.%
\par
This paragraph has some inline math, a Diophantine equation, \(x^2 + \doubler{y^2} = z^2\).  And some display math about infinite series: \begin{equation*}\sum_{n=1}^\infty\,\frac{1}{n^2} = \frac{\pi^2}{6}.\end{equation*}  Look at the XML source to see how \LaTeX{} macros are employed universally across all possible output formats.%
\typeout{************************************************}
\typeout{Section 1.3 Computation}
\typeout{************************************************}
\section[{Computation}]{Computation}\label{section-computation}
The following is a chunk of Sage code.  Your output format will dictate what you see next.  In print, you will see expected output.  In HTML you will have an executable, and editable, Sage Cell to work with.  In a SageMathCloud worksheet, you will be able to execute and edit the code with all the other features of SageMathCloud.  Note that if you include the expected output in your source, then you can test the example to verify that the behavior of Sage has not changed.%
\begin{lstlisting}[style=sageinput]
A = matrix(4,5, srange(20))
A.rref()
\end{lstlisting}
\begin{lstlisting}[style=sageoutput]
[ 1  0 -1 -2 -3]
[ 0  1  2  3  4]
[ 0  0  0  0  0]
[ 0  0  0  0  0]
\end{lstlisting}
%
%% A lineskip in table of contents as transition to appendices, backmatter
\addtocontents{toc}{\vspace{\normalbaselineskip}}
%
%
\appendix
%
\typeout{************************************************}
\typeout{Appendix A Hints and Solutions to Selected Exercises}
\typeout{************************************************}
\chapter[{Hints and Solutions to Selected Exercises}]{Hints and Solutions to Selected Exercises}\label{appendix-1}
\subsection*{1.1.1 Exercises 1.1 Linear Functions and models}
For problems 1-6, given two points in the (q,p) plane and a value q_0:Find the slope of the line determined by the points.Give the equation of the line determined by the points.Give the value of p predicted for q_0 by the line.\noindent\textbf{1.}\quad{} 
Points \((2,5)\) and \((6,17)\), with \(q_0=4\).
%
\par\smallskip
Find the slope and use the point-slope form%
\par\smallskip
\leavevmode%
\begin{enumerate}[label=(\alph*)]
\item\hypertarget{li-4}{}First find the slope: \(m=  \frac{\text{change in }p}{\text{change in }q}
=  \frac{17-5}{6-2}=\frac{12}{4}=3\)%
\item\hypertarget{li-5}{}Next we find the equation of the line. There are several ways to do this and two methods are outlined below.%
%
\begin{itemize}[label=\textbullet]
\item{}Method 1: use the point-slope equation: \(p-p_0=m (q-q_0)\).
We can choose either one of the points, so in this case we will find the line using the point \((q_0,p_0 )=(2,5)\). This gives the equation
\(p-5=3 (q-2)\).%
\par
Rewrite this as \(p=3q-1\)%
\item{}Method 2: use the slope- intercept equation \(p=m q+b\).
Use \((q,p)=(2,5)\) and \(m = 3\) and solve for \(b\):
\(5=3 (2)+b\).
And solving for \(b\) we have that \(b= -1\), and hence \(p=3q-1\)%
\end{itemize}
\item\hypertarget{li-8}{}Evaluate at the given point.  \(p(4)=3*4-1=11\)%
\end{enumerate}
\par\smallskip
\noindent\textbf{3.}\quad{} Points \((20,10)\) and \((40,5)\), with \(q_0=12\).
%
\par\smallskip
Just as in problem 1 we find the slope and then find the equation of the line.%
\leavevmode%
\begin{enumerate}[label=(\alph*)]
\item\hypertarget{li-9}{}First find the slope: \(m=  \frac{\text{change in }p}{\text{change in }q}
=  \frac{5-10}{40-20}=-\frac{5}{20}=-\frac{1}{4}\)%
\item\hypertarget{li-10}{}Using \(p=m (q-q_0)+p_0\) with \((q_0,p_0 )=(20, 10)\) and \(m = -\frac{1}{4}\), we get \(p=-\frac{1}{4}(q-20)+10\).  Solving for \(p\) we get \(p =-\frac{1}{4}q+15\)%
\item\hypertarget{li-11}{}Evaluate at the given point.  \(p(12)=-\frac{1}{4}(12)+15=12\)%
\end{enumerate}
\par\smallskip
\noindent\textbf{5.}\quad{} Points \((273,578)\) and \((412,6)\), with \(q_0=309\).
%
\par\smallskip
Just as in problem 1 we find the slope and then find the equation of the line.%
\leavevmode%
\begin{enumerate}[label=(\alph*)]
\item\hypertarget{li-12}{}First find the slope: \(m=  \frac{\text{change in }p}{\text{change in }q}
=  \frac{578-6}{273-412}=-\frac{5}{20}=-\frac{572}{139}\)%
\item\hypertarget{li-13}{}Using \(p=m (q-q_0)+p_0\) with \((q_0,p_0 )=(412, 6)\) and \(m = -\frac{572}{139}\), we get \(p=-\frac{572}{139}(q-412)+6\).  (We can combine the constant terms – the \(6\) and the \(-\frac{572}{139}*(-412)\), but leaving the equation in this form is acceptable.)%
\item\hypertarget{li-14}{}Evaluate at the given point.  \(p(309)=-\frac{572}{139}(309-412)+6
=-\frac{572}{139}(-103)+6=429\frac{119}{139}\)%
\end{enumerate}
\par\smallskip
\typeout{************************************************}
\typeout{Appendix B Notation}
\typeout{************************************************}
\chapter[{Notation}]{Notation}\label{appendix-2}
\typeout{************************************************}
\typeout{Introduction  }
\typeout{************************************************}
The following table defines the notation used in this book. Page numbers or references refer to the first appearance of each symbol.%
\begin{longtable}[l]{lp{0.60\textwidth}r}
\textbf{Symbol}&\textbf{Description}&\textbf{Page}\\[1em]
\endfirsthead
\textbf{Symbol}&\textbf{Description}&\textbf{Page}\\[1em]
\endhead
\multicolumn{3}{r}{(Continued on next page)}\\
\endfoot
\endlastfoot
\end{longtable}
%
\backmatter
%
%
%% The index is here, setup is all in preamble
\printindex
%
\cleardoublepage
\pagestyle{empty}
\vspace*{\stretch{1}}
\centerline{This book was authored and produced with \href{https://mathbook.pugetsound.edu}{MathBook XML}.%
}
\vspace*{\stretch{2}}
\end{document}